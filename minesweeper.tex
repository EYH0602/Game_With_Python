\documentclass[12pt]{article}
\usepackage[UTF8]{ctex}
\usepackage[margin=1in]{geometry}
\usepackage{listings,lstautogobble}
\usepackage{xcolor}

% code settings
\lstset{
    language=Python,
    basicstyle=\scriptsize\ttfamily,
    commentstyle=\ttfamily\itshape\color{gray},
    stringstyle=\ttfamily,
    showstringspaces=false,
    breaklines=true,
    frameround=ffff,
    frame=single,
    rulecolor=\color{black},
    autogobble=true
}

% title
\title{第7章 Minesweeper}
\author{Ethan He}
\date{}

\begin{document}
    \maketitle
    前面几张已经介绍过Pygame模块的基本使用方法,本章将会把重点放在实现扫雷游戏的Python语法和算法。
    在第6章 Connect 4中,我们介绍了矩阵(二维数组)的结构和基本算法,在这章则会接触矩阵的基本搜素算法。
    除此之外,本章将会介绍一些在应用方便的基础知识,比如应用广泛的JSON文件和正则表达式。%和匿名函数
    
    我们在Minesweeper游戏将要涉及的几个功能如下:
    \begin{itemize}
        \item 元组
        \item Pygame 通过鼠标的人机交互
        \item 文件处理
        \item JSON文件
        \item 正则表达式 Regular Expression
        % \item 匿名函数 Lambda Expression
        \item 二维数组的深度优先搜索
    \end{itemize}
    
    \section*{7.1 元组 Tuple}
        \subsection*{7.1.1 Tuple 基本语法}
            元组是一种常用的对象类型,很多高层编程语言都有它的身影。
            在C++中他叫作pair或是set,在Java中它叫做Tuple。元组的作用其实很简单:表示一组\textbf{相关的}数据。
            等等,表示一组数据?为什么这句话听起来与第6章学习的list如此相似?
            是的,Tuple这种数据类型的应用场景和方式与list的确有很多相似的地方。来看看语法:
            \begin{lstlisting}
                fruit_tuple = ('apple', 'banana', 'orange') # a tuple of fruits
                fruit_list = ['apple', 'banana', 'orange']  # a list of fruits
            \end{lstlisting}

            Python作为一种高级语言,一个元组所包含的元素类型可以不一样, 也可以有多种创建方法。
            在创建元组的时候,Python允许我们使用()来表示这组数据的类型为元组,就像[]表示一组数据为列表一样。
            与此同时,Python同样允许我们不写(),同样表示元组。
            然而,作者极其不推荐这种写法,因为很多时候这种创建方式的表达意义并不明确,会带来很差的可读性。
            要时刻记住,写代码的首要目的是让人能看懂,其次才是让计算机运行
            \begin{lstlisting}
                tup1 = ('apple', 50, 'banana', 16.7)    # 包含不同类型的元素
                tup2 = 'apple', 50, 'banana', 16.7      # 不是使用()创建
                tup3 = ()   # 创建空元组
                tup4 = ('delicious',)   # 创建只有一个元素的元组,有逗号
                tup5 = ('delicious')    # 创建只有一个元素的元组,无逗号

                print(tup1)
                print(tup2)
                print(tup3)
                print(tup4)
                print(tup5)
            \end{lstlisting}

            依次打印tup1-tup4, 看看这段代码的运行结果是什么。从运行结果可以看出,tup4和tup5的打印结果
            有很大的区别。如果元素只有一个元素,请务必记住要在元素后加上都好',',否则Python解释器会把你的
            意思理解为字符串,如同tup5的运行结果:
            \begin{lstlisting}
                ('apple', 50, 'banana', 16.7)
                ('apple', 50, 'banana', 16.7)
                ()
                ('delicious',)
                delicious
            \end{lstlisting}

            在了解如何创建元组后,我们需要做的就是访问元组。
            访问元组元素的方式与列表类似,通过\lstinline{objName[index]}的方式来表示元组中的元素。
            \begin{lstlisting}
                tup = ('apple', 'banana', 'orange', 'peach') # a tuple of fruits
                print(tup)
                print(tup[0])   # apple
                print(tup[-2])  # orange
                print(tup[1:])  # ('banana', 'orange', 'peach')

                for fruit in tuo:
                    print(fruit)
            \end{lstlisting}
            运行结果如下:
            \begin{lstlisting}
                ('apple', 'banana', 'orange')
                apple
                orange
                ('banana', 'orange', 'peach')
                apple
                banana
                orange
                peach
            \end{lstlisting}

            除了简单的创建和访问之外,元组类还支持一些列的基础计算。
            Python允许我们使用‘+’操作符来合并两个元组,'*'操作符来复制元组的元素:
            \begin{lstlisting}
                tup1 = (1, 'a', 2)
                tup2 = ('b', 3, 'c')
                tup3 = tup1 + tup2
                print(tup3) # (1, 'a', 2, 'b', 3, 'c')

                tup4 = ('abc',)*3
                print(tup4) # ('abc', 'abc', 'abc')
            \end{lstlisting}

            元组最为Python的一个类,当然要有成员函数。
            Python给元组类定义了五个最常用的成员函数:
            \begin{itemize}
                \item len(tuple)
                \item max(tuple)
                \item min(tuple)
                \item tuple(list)
            \end{itemize}
            \begin{lstlisting}
                tup1 = (1, 2, 3)
                tup2 = (1, 2, 3)
                list1 = [4, 5, 6]
                
                print(len(tup1))    # 3
                print(max(tup1))    # 3
                print(min(tup1))    # 1
                print(tuple(list1)) # (4, 5, 6)
            \end{lstlisting}

        \subsection*{7.1.2 Tuple与List的相同点}


        \subsection*{7.1.2 Tuple与List的区别}





    
    \section*{7.2 Pygame 通过鼠标的人机交互}
        在上一章,我们学习了Pygame模块的基础功能,比如设置游戏屏幕,键盘交互,绘制图形等。
        在这一章,因为扫雷不能用只用键盘来玩,我们将要学习如何通过鼠标点击与计算机交互。

    \section*{7.3 文件处理}
    \section*{7.4 JSON}
    \section*{7.5 正则表达式 Regular Expression}
    \section*{7.6 二维数组的深度优先搜索}
    \section*{7.7 Minesweeper 游戏编写}


        
\end{document}
